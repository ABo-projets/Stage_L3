\documentclass{beamer}

\usetheme{CambridgeUS}
\usecolortheme{seahorse}

\usepackage{geometry}

\usepackage{algorithm2e}
\usepackage{graphicx}
\usepackage{tabularx}
\usepackage{fourier}
\usepackage{hyperref}
\usepackage{multirow}
\usepackage{caption}
\usepackage{subcaption}
\usepackage{amsmath}
\usepackage{float}





	\usepackage{rotating}	%% introduce sideways env
\usepackage{tikz}
\usepackage{amssymb}

%% Public TikZ libraries
\usetikzlibrary{positioning}
\usetikzlibrary{shapes}
\usetikzlibrary{calc,cipher,sponge}


\usetikzlibrary{arrows}			%% Include regular arrows
\usetikzlibrary{arrows.meta}	
%\usepgflibrary{myarrows.new}		%% Include custom arrows
%\usetikzlibrary{mycrypto.symbols}


\tikzset{XOR/.style={draw,
		circle,
		append after command={
			[shorten >=\pgflinewidth, shorten <=\pgflinewidth,]
			(\tikzlastnode.north) edge[] (\tikzlastnode.south)
			(\tikzlastnode.east) edge[] (\tikzlastnode.west)
		},
	},
}

\tikzset{edge/.style={
		-latex new, 
		arrow head=6pt, 
		%						thick
	},
}

\tikzset{line/.style={
		%						thick,
		draw, 
		%						-latex',
		shorten <=1bp,
		shorten >=1bp,
	}
}

\newcommand{\AEheight}{0.75cm}
\newcommand{\AEwidth}{1.5cm}

\tikzset{AE_P/.style={
		rectangle,
		thick,
		draw,
		fill=yellow!20,
		inner sep = 0pt,
		outer sep = 0pt,
		text width=1cm, 
		minimum height=4.5cm,
		text centered,
		%		anchor=north,
	}	
}

\tikzset{AE_vert/.style={
		rectangle,
		thick,
		draw,
		fill=blue!20,
		inner sep = 0pt,
		outer sep = 0pt,
		text width=\AEheight, 
		minimum height=\AEwidth,
		text centered,
		%		anchor=north,
	}
}

\tikzset{AE_msg/.style={
		rectangle,
		thick,
		draw,
		fill=blue!20,
		inner sep = 0pt,
		outer sep = 0pt,
		text width=\AEwidth,
		minimum height=\AEheight,
		text centered,
		%		anchor=north,
	}
}

\tikzset{AE_ctx/.style={
		rectangle,
		thick,
		draw,
		fill=green!20,
		inner sep = 0pt,
		outer sep = 0pt,
		text width=\AEwidth, 
		minimum height=\AEheight,
		text centered,
		%		anchor=north,
	}
}

\tikzset{AE_key/.style={
		rectangle,
		thick,
		draw,
		fill=red!20,
		inner sep = 0pt,
		outer sep = 0pt,
		text width=\AEwidth, 
		minimum height=\AEheight,
		text centered,
		%		anchor=north,
	}
}

\tikzset{AE_tag/.style={
		rectangle,
		thick,
		draw,
		fill=green!20,
		inner sep = 0pt,
		outer sep = 0pt,
		text width=\AEwidth, 
		minimum height=\AEheight,
		text centered,
		%		anchor=north,
	}
}





\usepackage{xcolor}
\usepackage{listings}

\definecolor{mGreen}{rgb}{0,0.6,0}
\definecolor{mGray}{rgb}{0.5,0.5,0.5}
\definecolor{mPurple}{rgb}{0.58,0,0.82}
\definecolor{backgroundColour}{rgb}{0.95,0.95,0.92}

\lstdefinestyle{CStyle}{
	backgroundcolor=\color{backgroundColour},   
	commentstyle=\color{mGreen},
	keywordstyle=\color{magenta},
	numberstyle=\tiny\color{mGray},
	stringstyle=\color{mPurple},
	basicstyle=\tiny,
	breakatwhitespace=false,         
	breaklines=true,                 
	captionpos=b,                    
	keepspaces=true,                 
	numbers=left,                    
	numbersep=5pt,                  
	showspaces=false,                
	showstringspaces=false,
	showtabs=false,                  
	tabsize=2,
	language=C
}

\author{Alexane Boldo}

\institute{OCIF, IRISA}

\title{Side-channel attacks on Ascon's S-box}

\date{May-July 2025}

\setbeamertemplate{page number in head/foot}[appendixframenumber]

\begin{document}
	\begin{frame}
		\maketitle\small
		{\centering\itshape Supervisor: Hélène Le Bouder\par}
	\end{frame}
	
	\begin{frame}
		\frametitle{Introduction}
		\textbf{Side-Channel Attacks (SCA):} observation of computation time, power consumption, electromagnetic radiation, ... to discover a secret
		
		$\newline$
		
		\textbf{Goal:} Study the leaks from the winner for lightweight cryptography Ascon to theorize a SCA attack
	\end{frame}
	
%	\AtBeginSection[]
%	{
%		\begin{frame}
%			\frametitle{Table of Contents}
%			\tableofcontents[currentsection]
%		\end{frame}
%	}
	
	\begin{frame}
		\frametitle{Table of Contents}
		\tableofcontents
	\end{frame}
	
	\section{Ascon-AEAD}
	\subsection{Presentation}
	\begin{frame}
		\frametitle{What is Ascon-AEAD?}
		\textbf{Authenticated Encryption with Associated Data (AEAD)}: encrypt, check authentication of content and associated data
		
		\begin{figure}
			\centering
			\documentclass{standalone}
\usepackage{rotating}	%% introduce sideways env
\usepackage{tikz}
\usepackage{amssymb}

%% Public TikZ libraries
\usetikzlibrary{positioning}
\usetikzlibrary{shapes}
\usetikzlibrary{calc}


\usetikzlibrary{mycrypto.symbols}

%\usepackage{fmtcount}

\begin{document}

\begin{tikzpicture}[node distance = 1cm]

\node[AE_E] (aead) {AEAD};

\node[left = of $(aead.north west)!3/4!(aead.south west)$] (m) {$M$};
\node[left = of $(aead.north west)!1/4!(aead.south west)$] (a) {$A$};

\node[above = of $(aead.north west)!1/4!(aead.north east)$] (k) {$K$};
\node[above = of $(aead.north west)!3/4!(aead.north east)$] (n) {$N$};

\node[right = of $(aead.north east)!1/4!(aead.south east)$] (c) {$C$};
\node[right = of $(aead.north east)!3/4!(aead.south east)$] (t) {$T$};


\draw[edge] (m) -- (aead.west|-m);
\draw[edge] (a) -- (aead.west|-a);

\draw[edge] (k) -- (aead.north-|k);
\draw[edge] (n) -- (aead.north-|n);

\draw[edge] (aead.east|-c) -- (c);
\draw[edge] (aead.east|-t) -- (t);

\end{tikzpicture}
\end{document}
			\caption{AEAD algorithm from \cite{cours_crypto}}
		\end{figure}
		
	\end{frame}
	
	\begin{frame}
		\frametitle{Ascon's State}
		\begin{tabular}{ccccccccl}
			byte 0&byte 1&byte 2&byte 3&byte 4&byte 5&byte 6&byte 7&\\
			\cline{1-8}
			\multicolumn{8}{|c|}{IV}&$S_0$\\
			\cline{1-8}
			\multicolumn{8}{|c|}{first half of K}&$S_1$\\
			\cline{1-8}
			\multicolumn{8}{|c|}{Second half of K}&$S_2$\\
			\cline{1-8}
			\multicolumn{8}{|c|}{first half of N}&$S_3$\\
			\cline{1-8}
			\multicolumn{8}{|c|}{Second half of N}&$S_4$\\
			\cline{1-8}
		\end{tabular}
	\end{frame}
	
	\subsection{Encryption and decryption phases}
	\begin{frame}
		\frametitle{Encryption and decryption phases}
		
		4 phases: Initialization, Associated data process, plaintext/ciphertext process, Finalization
		
		\begin{figure}
			\centering
			\resizebox{350pt}{80pt}{
				\resizebox{250pt}{50pt}{
	
\begin{tikzpicture}[node distance = 1cm and 0.75cm]

% init
\node[AE_P, minimum height = 3.75cm] (pi) {$p^{12}$};

\node[AE_vert, fill = gray!20, left = of pi.north west, anchor = north east, minimum height = 0.75cm] (iv) {$IV$}; 

\node[AE_vert, fill = red!20, anchor = north] (k)  at (iv.south) {$K$}; 
\node[AE_vert, fill = blue!20, left = of pi.south west, anchor = south east] (n) {$N$}; 

\coordinate (helpsouth) at ($(pi.south east)!1/6!(pi.north east)$);
\coordinate (helpnorth) at ($(pi.south east)!5/6!(pi.north east)$);

\draw[-{Latex[length=2mm]}] (iv.east|-helpnorth) -> (helpnorth-|pi.west) node[midway] {$/$} node[midway, below=0.2cm] {\small $64$};
\draw[-{Latex[length=2mm]}] (k.east|-pi.west) -- (pi.west) node[midway] {$/$} node[midway, below=0.2cm] {\small $128$};
\draw[-{Latex[length=2mm]}] (n.east|-helpsouth) -- (helpsouth-|pi.west) node[midway] {$/$} node[midway, below=0.2cm] {\small $128$};

\node[XOR, right = of pi.east|-helpsouth] (xkinit) {};
\node[AE_key, anchor = west, below = of xkinit, path picture={
      \fill[gray!20] (path picture bounding box.north) rectangle (path picture bounding box.south west);
    }] (kinit) {$0^* || K$};

\draw[-{Latex[length=2mm]}] (pi.east|-helpsouth) -- (xkinit) node[midway] {$/$} node[midway, below=0.2cm] {\small $256$}; 
\draw[-{Latex[length=2mm]}] (kinit.north-|xkinit) -- (xkinit); 



% 1er tour AD

\node[XOR, right = of kinit.east|-helpnorth] (xa1) {};

\node[AE_msg, above = of xa1] (a1) {$A_1$};

\draw[-{Latex[length=2mm]}] (a1) -- (xa1) node[midway] {$/$} node[midway, left=0.2cm] {\small $64$};
\draw[line] (pi.east|-xa1) -- (helpnorth-|xkinit) node[midway] {$/$} node[midway, below=0.2cm] {\small $64$};
\draw[-{Latex[length=2mm]}] (pi.east|-xa1)  -- (xa1);


% 2er tour AD


\node[AE_P, minimum height = 3.75cm, right = of xa1|-pi] (pa1) {$p^{8}$};

\draw[-{Latex[length=2mm]}] (xkinit) -- (pa1.west|-xkinit);
\draw[-{Latex[length=2mm]}] (xa1) -- (pa1.west|-xa1);

\node[right = of pa1.east|-helpnorth, inner sep = 10pt] (cdotnortha) {$\cdots$};
\node[right = of pa1.east|-helpsouth, inner sep = 10pt] (cdotsoutha) {$\cdots$};

\node[XOR, right = of cdotnortha] (xa2) {};
\node[AE_msg, above = of xa2, path picture={
      \fill[gray!20] (path picture bounding box.north) rectangle (path picture bounding box.south east);
    }] (a2) {$A_{s}||10^*$};


\draw[line] (pa1.east|-cdotnortha) -- (cdotnortha);
\draw[line] (pa1.east|-cdotsoutha) -- (cdotsoutha);
\draw[-{Latex[length=2mm]}] (cdotnortha) -- (xa2);
\draw[-{Latex[length=2mm]}] (a2) -- (xa2) node[midway] {$/$} node[midway, left=0.2cm] {\small $64$};


% fin AD

\node[AE_P, minimum height = 3.75cm, right = of xa2|-pa1] (pa2) {$p^{8}$};

\node[XOR, right = of pa2.east|-helpsouth] (xdom) {};
\node[AE_key, fill = gray!20, below = of xdom] (dom) {$0^*1$};

\draw[-{Latex[length=2mm]}] (xa2) -- (pa2.west|-xa2);
\draw[-{Latex[length=2mm]}] (cdotsoutha) -- (pa2.west|-helpsouth);

\draw[-{Latex[length=2mm]}] (pa2.east|-xdom) -- (xdom);
\draw[-{Latex[length=2mm]}] (dom.north-|xdom) -- (xdom);



% 1er tour chiffrement

\node[XOR, right = of dom.east|-helpnorth] (x1) {};
\node[AE_msg, above = of x1] (m1) {$M_1$};
\node[AE_ctx, right = of m1] (c1) {$C_1$};

\draw[-{Latex[length=2mm]}] (pa2.east|-x1) -- (x1);
\draw[-{Latex[length=2mm]}] (m1) -- (x1) node[midway] {$/$} node[midway, left=0.2cm] {\small $64$};
\draw[-{Latex[length=2mm]}] (x1) -| (c1)node[near end] {$/$} node[near end, left=0.2cm] {\small $64$};

\node at (x1-|c1) {$\bullet$};

% 2e tour chiffrement

\node[AE_P, minimum height = 3.75cm, right = of c1|-pa2] (p1) {$p^{8}$};

\draw[-{Latex[length=2mm]}] (x1) -- (x1-|p1.west);
\draw[-{Latex[length=2mm]}] (xdom) -- (p1.west|-helpsouth);

\node[right = of p1.east|-helpnorth, inner sep = 10pt] (cdotnorth) {$\cdots$};
\node[right = of p1.east|-helpsouth, inner sep = 10pt] (cdotsouth) {$\cdots$};

\node[XOR, right = of cdotnorth] (x2) {};
\node[AE_msg, above = of x2] (m2) {$M_{t-1}$};
\node[AE_ctx, right = of m2] (c2) {$C_{t-1}$};


\draw[line] (p1.east|-cdotnorth) -- (cdotnorth);
\draw[line] (p1.east|-cdotsouth) -- (cdotsouth);

\draw[-{Latex[length=2mm]}] (cdotnorth) -- (x2);
\draw[-{Latex[length=2mm]}] (m2) -- (x2) node[midway] {$/$} node[midway, left=0.2cm] {\small $64$};
\draw[-{Latex[length=2mm]}] (x2) -| (c2) node[near end] {$/$} node[near end, left=0.2cm] {\small $64$};

\node at (x2-|c2) {$\bullet$};


% fin chiffrement

\node[AE_P, minimum height = 3.75cm, right = of c2|-p1] (p2) {$p^{8}$};

\draw[-{Latex[length=2mm]}] (x2) -- (x2-|p2.west);
\draw[-{Latex[length=2mm]}] (cdotsouth) -- (p2.west|-cdotsouth);

\node[XOR, right = of p2.east|-x1] (x3) {};
\node[AE_msg, above = of x3, path picture={
      \fill[gray!20] (path picture bounding box.north) rectangle (path picture bounding box.south east);
    }] (m3) {$M_{t}||10^*$};
\node[AE_ctx, right = of m3, text width = 0.75cm] (c3) {$C_{t}$};

\draw[-{Latex[length=2mm]}] (p2.east|-x3) -- (x3);
\draw[-{Latex[length=2mm]}] (m3) -- (x3) node[midway] {$/$} node[midway, left=0.2cm] {\small $64$};
\draw[-{Latex[length=2mm]}] (x3) -| (c3) node[near end] {$/$} node[near end, left=0.2cm] {\small $|M_t|$};

\node at (x3-|c3) {$\bullet$};


% finalisation

\node[XOR, right = of c3.east|-helpsouth] (xkf) {};

\node[AE_P, minimum height = 3.75cm, right = of xkf.east|-p2] (pf) {$p^{12}$};

\node[AE_key, below = of xkf, path picture={
      \fill[gray!20] (path picture bounding box.north) rectangle (path picture bounding box.south east);
    }] (kf) {$K||0^*$};

\draw[-{Latex[length=2mm]}] (x3) -- (x3-|pf.west);
\draw[-{Latex[length=2mm]}] (p2.east|-xkf) -- (xkf);
\draw[-{Latex[length=2mm]}] (xkf|-kf.north) -- (xkf);
\draw[-{Latex[length=2mm]}] (xkf) -- (xkf-|pf.west);

\node[XOR, right = of pf.east|-xkf] (xt) {};
\node[AE_key, below = of xt] (kt) {$K$};
\node[AE_tag] at (c3-|xt) (t) {$T$};

\draw[-{Latex[length=2mm]}] (pf.east|-xt) -- (xt) node[midway] {$/$} node[midway, below=0.2cm] {\small $128$};
\draw[-{Latex[length=2mm]}] (kt) -- (xt);
\draw[-{Latex[length=2mm]}] (xt) -- (t);


% separations

\coordinate (sepinit) at ($(kinit.east)!1/2!(a1.west)$);
\draw[line, dashed] (kinit.south-|sepinit) -- (a1.north-|sepinit);

\coordinate (sepa) at ($(dom.east)!1/2!(m1.west)$);
\draw[line, dashed] (dom.south-|sepa) -- (m1.north-|sepa);

\coordinate (sepf) at ($(c3.east)!1/2!(kf.west)$);
\draw[line, dashed] (kf.south-|sepf) -- (c3.north-|sepf);

\end{tikzpicture}
}
			}
			\caption{Ascon-AEAD mode, from \cite{cours_crypto}}
		\end{figure}	
	\end{frame}
	
	\subsection{The permutation}
	\begin{frame}
		\frametitle{Ascon's permutation(s)}
		$p = p_L \circ p_S \circ p_C$\\
		
		\begin{columns}[T]
			\column{0.47\textwidth}
			\begin{figure}[h]
				\centering
				\includegraphics[width=0.6\linewidth]{img_files/sbox_computation}
				\caption{S-box computation for the first byte of each word}
				\label{fig:comp}
			\end{figure}
			
			\column{0.47\textwidth}
			\begin{figure}
				\resizebox*{150pt}{90pt}{
				\input{ascon-sbox}
				}
				\caption{Circuit to compute the S-box, from \protect \footnotemark}
				\label{circuit_sbox}
			\end{figure}
			
			\footnotetext{\url{https://extgit.isec.tugraz.at/meichlseder/tikz}}

		\end{columns}
	\end{frame}
	
	\subsection{An analysis of the S-box}
	\begin{frame}
		\frametitle{Table linking the output of the S-box and the key}
		\begin{figure}[h]
			\centering
			\begin{tabular}{|c|c|}
				\hline
				$(n_0^j,n_1^j,IV^j)$&$S_4^j$\\
				\hline\hline
				$(0,0,0)$&$k_0^j$\\
				\hline
				$(0,0,1)$&$0$\\
				\hline
				$(0,1,0)$&$1$\\
				\hline
				$(0,1,1)$&$1 \oplus k_0^j$\\
				\hline
				$(1,0,0)$&$1 \oplus k_0^j$\\
				\hline
				$(1,0,1)$&$1$\\
				\hline
				$(1,1,0)$&$0$\\
				\hline
				$(1,1,1)$&$k_0^j$\\
				\hline
			\end{tabular}
			\caption{Link between $k_0^j$ and $S_4^j$ depending on $IV$ and $N$, from \cite{these}}
			\label{link_k_s4}
		\end{figure}
	\end{frame}
	
	\section{A side-channel attack}
	\subsection{Hardware}
	\begin{frame}
		\frametitle{ChipWhisperer-Lite}
		\begin{figure}[h]
			\raggedright
			\includegraphics[width=0.9\textwidth]{img_files/cwlite_basic1}
			\caption{ChipWhisperer Lite board, from \cite{cwdoc}}
			\label{fig:cw}
		\end{figure}
	\end{frame}
	
	\subsection{CPA attacks}
	\begin{frame}
		\frametitle{Steps for a CPA attack}
		\begin{itemize}
			\item Compute the algorithm multiple times to gain traces
			\item Find model for the consumption
			\item Deduce the hypothesis that correlates best
		\end{itemize}
		\begin{figure}[H]
			\includegraphics[width=0.5\textwidth]{img_files/corr_aes}
			\caption{CPA on another encryption standard}
		\end{figure}
	\end{frame}

	\subsection{Analysis}
	\begin{frame}
		\frametitle{Analyses done}
		\begin{itemize}
			\item{Finding the best model}
				\begin{itemize}
					\item{Vertical vs horizontal}
					\item{HW vs value}
				\end{itemize}
			\item{Attack: finding the vertical output and deduct the key}
		\end{itemize}
	\end{frame}
	
	
	\section{Results}
	\begin{frame}
		\frametitle{Results vertical vs horizontal and HW vs value}
		\begin{columns}[T]
			\column{0.47\textwidth}
			\begin{figure}[h]
				\centering
				\includegraphics[scale=0.25]{img_files/h_and_v_one_byte}
				\caption{Mutual information for the horizontal and the vertical value}
				\label{hvval}
			\end{figure}
			\column{0.47\textwidth}
			\begin{figure}
				\centering
				\includegraphics[scale=0.25]{img_files/vertical_one_bit}
				\caption{Mutual information between power consumption and HW or value}
				\label{vHW}
			\end{figure}
		\end{columns}
	\end{frame}
	
%	\begin{frame}
%		\frametitle{Results distinguisher}
%		\begin{figure}
%			\centering
%			\includegraphics[scale=0.3]{img_files/corr_vs_MI_hHW}
%			\caption{Mutual information and absolute Pearson correlation for a horizontal attack on the reference implementation}
%			\label{corvsMI}
%		\end{figure}
%	\end{frame}
	
	\begin{frame}
		\frametitle{Results attack}
		\begin{figure}[h]
			\centering
			\includegraphics[scale=0.3]{img_files/nonces_alea}
			\caption{Mutual information between the Hamming weight of the outputs and the power consumption, for each of the possible outputs for the first nonce}
			\label{all_alea}
		\end{figure}
	\end{frame}
	
	\section{Conclusion}
	\begin{frame}
		\frametitle{Conclusion}
		\begin{itemize}
			\item Good leaks compared to random values
			\item Though apparent weaknesses, unsuccessful attempts
			\item Not enough randomness with false key hypotheses
			\item Leads to follow: belief propagation
		\end{itemize}
	\end{frame}
	
	
	\appendix
	\bibliographystyle{IEEEtran} 
	\bibliography{IEEEabrv,refs}
	
	\section{Equations for linking the output of the S-box to the key}
	\begin{frame}
		\frametitle{Finding this table (1)}
		\begin{gather*}
			S_4^j = n_o^j \oplus n_1^j \oplus k_0^j \times (1 \oplus IV^j \oplus n_1^j)\\
			S _4^j =\left \{	
			\begin{array}{ll}
				k_0^j \times (1 \oplus IV^j) & if\ (n_0^j,n_1^j)=(0,0)\\
				k_0^j \times IV^j & if\ (n_0^j,n_1^j)=(1,1)\\
				1 \oplus k_0^j \times IV^j & if\ (n_0^j,n_1^j)=(0,1)\\
				1 \oplus k_0^j \times (1 \oplus IV^j) & if\ (n_0^j,n_1^j)=(1,0)\\
			\end{array}
			\right.
		\end{gather*}
	\end{frame}
	
	\begin{frame}
		\frametitle{Finding this table (2)}
		\noindent Then if $IV^j = 0$: 
		$$S _4^j =\left \{	
		\begin{array}{ll}
			k_0^j& if\ (n_0^j,n_1^j)=(0,0)\\
			0& if\ (n_0^j,n_1^j)=(1,1)\\
			1& if\ (n_0^j,n_1^j)=(0,1)\\
			1 \oplus k_0^j& if\ (n_0^j,n_1^j)=(1,0)\\
		\end{array}
		\right.$$
	\end{frame}
	
	\begin{frame}
		\frametitle{Finding this table (3)}
		\noindent Otherwise, if $IV^j = 1$:
		$$S _4^j =\left \{	
		\begin{array}{ll}
			0& if\ (n_0^j,n_1^j)=(0,0)\\
			k_0^j& if\ (n_0^j,n_1^j)=(1,1)\\
			1 \oplus k_0^j& if\ (n_0^j,n_1^j)=(0,1)\\
			1& if\ (n_0^j,n_1^j)=(1,0)\\
		\end{array}
		\right.$$
	\end{frame}
	
	\section{Complementary graphs}
	\begin{frame}
		\frametitle{Complementary graph (1)}
		\begin{figure}[H]
			\centering
			\includegraphics[scale=0.3]{img_files/vertical_one_byte}
			\caption{Mutual information between power consumption and Hamming weight of the concatenation of the first bit of each of the word of $S$ and its value like \ref{vHW} but for random nonces}
			\label{vHW&val}
		\end{figure}
	\end{frame}
	
	\begin{frame}
		\frametitle{Complementary graph (2)}
		\begin{figure}[H]
			\centering
			\includegraphics[scale=0.3]{img_files/HWalea}
			\caption{Mutual information between power consumption and vertical HW or random possible HW}
			\label{HWalea}
		\end{figure}
	\end{frame}
	
	\section{Main code} \lstinputlisting[language=C,style=Cstyle,caption=Implementation for ascon.c]{code_files/ascon.c}
	
	\lstinputlisting[language=C,style=Cstyle,caption=Implementation for ascon.h]{code_files/ascon.h}
	
	\lstinputlisting[language=C,style=Cstyle,caption=Implementation for permutation.c]{code_files/permutation.c}
	
	\lstinputlisting[language=C,style=Cstyle,caption=Implementation for permutation.h]{code_files/permutation.h}
	
	\lstinputlisting[language=C,style=Cstyle,caption=Implementation for main.c]{code_files/main.c}
	
	\lstinputlisting[language=Python,style=Cstyle,caption=Implementation for trace capture for the ChipWhisperer]{code_files/traces_sca.py}
	
	\lstinputlisting[style=CStyle, caption=Analysis in Julia of the traces following the established attack]{code_files/decryption_analysis_cpa.jl}
	

\end{document}